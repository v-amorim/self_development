\documentclass[
	% -- opções da classe memoir --
	12pt,		% tamanho da fonte
	%openright,	% capítulos começam em pág ímpar (insere página vazia caso preciso)
	oneside,	% para impressão em verso e anverso. Oposto a oneside
	a4paper,	% tamanho do papel.
	% -- opções da classe abntex2 --
	chapter=TITLE,		% títulos de capítulos convertidos em letras maiúsculas
	%section=TITLE,		% títulos de seções convertidos em letras maiúsculas
	%subsection=TITLE,	% títulos de subseções convertidos em letras maiúsculas
	%subsubsection=TITLE,% títulos de subsubseções convertidos em letras maiúsculas
	% -- opções do pacote babel --
	english,	% idioma adicional para hifenização
	brazil		% o último idioma é o principal do documento
]{abntex2}



%----------------------------------------------------------------------------------------
% Definição dos Packages
%----------------------------------------------------------------------------------------

\usepackage{silence}                            % Pra pular erros bestas do glossario
\WarningFilter{glossaries}{Overriding \printglossary}
\WarningFilter{glossaries}{Overriding `theglossary'}

\usepackage[subentrycounter,seeautonumberlist,nonumberlist=true]{glossaries}
\usepackage[T1]{fontenc}	                    % Selecao de codigos de fonte. Afeta separação de sílabas
\usepackage[brazilian,hyperpageref]{backref}    % Paginas com as citações na bibl
\usepackage[alf]{abntex2cite}                   % Citações padrão ABNT
\usepackage{lastpage}		                    % Usado pela ficha catalografica
\usepackage{indentfirst}	                    % Indenta o primeiro paragrafo de cada seção
\usepackage{microtype} 		                    % Para melhorias de justificação
\usepackage{bibentry}   	                    % para inserir refs. bib. no meio do texto
\usepackage{listings}                           % Define as listas (numeradas)
\usepackage{tikzsymbols}                        % Simbolos matemáticos (usado pra colocar o logo)
\usepackage{textcomp, gensymb}                  % Pra poder usar o º (\degree)
\usepackage{fontspec}                           % Pacote para a fonte Arial
%\usepackage{helvet}                             % Usado para definir a fonte em Helvetica
%\usepackage[utf8]{inputenc}	                 % Codificacao do documento (conversão automática dos acentos)

%----------------------------------------------------------------------------------------
% Execução dos Packages
%----------------------------------------------------------------------------------------
\citebrackets{[}{]}
\setmainfont{Arial} % Define a fonte em Arial
%\renewcommand{\familydefault}{\sfdefault}       % Define em Helvetica

\makeindex                                      % compila o indice

\usepackage{color}

\definecolor{keywordstyle}{RGB}{0,0,255}
\definecolor{stringstyle}{RGB}{184,21,21}
\definecolor{commentstyle}{RGB}{0,128,0}

\lstset{
	basicstyle=\ttfamily\tiny,
	keywordstyle=\color{keywordstyle},
	stringstyle=\ttfamily\color{stringstyle},
	commentstyle=\color{commentstyle},
	backgroundcolor=\color{gray!5},
	extendedchars=true,
	breaklines=true,
	frame=tb,
	showstringspaces=false,
	numbers=left,
	numberstyle=\tiny
}
% Configurações do pacote backref
\renewcommand{\backrefpagesname}{Citado na(s) página(s):~}
\renewcommand{\backref}{}                       % Texto padrão antes do número das páginas
\renewcommand*{\backrefalt}[4]{                 % Define os textos da citação
	\ifcase #1

	\or
		Citado na página #2.
	\else
		Citado #1 vezes nas páginas #2.
	\fi}

% Espaçamentos entre linhas e parágrafos
\linespread{1.5}                                % Espaçamento da linha
\setlength{\parindent}{1.3cm}                   % O tamanho do parágrafo
\setlength{\parskip}{0.2cm}                     % Controle do espaçamento entre um parágrafo e outro
