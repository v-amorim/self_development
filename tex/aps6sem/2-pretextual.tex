% Seleciona o idioma do documento (conforme pacotes do babel)
%\selectlanguage{english}
\selectlanguage{brazil}

% Retira espaço extra obsoleto entre as frases.
\frenchspacing

\newpage

% ==============================================
% ELEMENTOS PRÉ-TEXTUAIS
% ==============================================
\pretextual

% ----------------------------------------------
% Capa
% ----------------------------------------------
%\imprimircapa
% Capa personalizada sem o uso de \imprimircapa
\begin{capa}
	\begin{center}
		\begin{minipage}{1\textwidth}
			\large\centering\makebox[\textwidth]{\includegraphics[scale=1]{imagens/Logo UNIP.png}}
		\end{minipage}
	\end{center}
	\begin{center}
		\LARGE\textbf{UNIP -- UNIVERSIDADE PAULISTA\\}
		\LARGE {Curso de Ciência da Computação}\\
		\vfill
		\ABNTEXchapterfont\Large\textbf{\MakeUppercase{ATIVIDADES PRÁTICAS SUPERVISIONADAS - APS}}
		\\\small{DESENVOLVIMENTO DE UMA FERRAMENTA PARA COMUNICAÇÃO EM REDE}
		\vfill
		\normalsize{
			André Ademir de Sousa Oliveira - N629630\\
			Daniel Chrispim Domingos - F263614\\
			Gabriel de Paula Seki - N6853H6\\
			Luiz Guilherme Davies Lencioni - N667125\\
			Marcos Vinicius Restani Avanzini - G13HJB9\\
			Vinícius Amorim - N641JC5\\
		}
		\vfill
		São José dos Campos, \today
	\end{center}
\end{capa}

% ----------------------------------------------
% Folha de rosto
% ----------------------------------------------
% folha de rosto personalizada sem uso de \imprimirfolhaderosto
\makeatletter
\renewcommand{\folhaderostocontent}{
	\begin{center}
		\begin{center}
			\begin{minipage}{1\textwidth}
				\large\centering\makebox[\textwidth]{\includegraphics[scale=1]{imagens/Logo UNIP.png}}
			\end{minipage}
		\end{center}

		\vspace*{\fill}%\vspace*{\fill}
		\begin{center}
			\ABNTEXchapterfont\Large\textbf{\MakeUppercase{ATIVIDADES PRÁTICAS SUPERVISIONADAS - APS}}
			\\\small{DESENVOLVIMENTO DE UMA FERRAMENTA PARA COMUNICAÇÃO EM REDE}
		\end{center}
		\vspace*{\fill}

		\hspace{.45\textwidth}
		\begin{minipage}{.5\textwidth}
			\SingleSpacing
			{Atividades Práticas Supervisionadas do 5\degree\ Semestre do Curso de Ciência da Computação da \textbf{Universidade Paulista UNIP.}}
		\end{minipage}%
		\vspace*{\fill}


		\hspace{.45\textwidth}
		\begin{minipage}{.5\textwidth}
			{\textbf{Coordenador:} Prof. Fernando A. Gotti}%
		\end{minipage}%


		\hspace{.45\textwidth}
		\begin{minipage}{.5\textwidth}
			{\textbf{Prof. Responsável:} André Yoshimi Kusumoto}%
		\end{minipage}%


		\vspace*{\fill}
		%{\abntex@ifnotempty{\imprimirinstituicao}{\imprimirinstituicao\vspace*{\fill}}}

		São José dos Campos, \today
	\end{center}
}
\makeatother

% Folha de rosto (o * indica que haverá a ficha bibliográfica)

\imprimirfolhaderosto

% ||||||||||||||||||||||||||||||||||||||||||||||
% RESUMOS
% ||||||||||||||||||||||||||||||||||||||||||||||

% ----------------------------------------------
% Resumo em português
% ----------------------------------------------
% Importante: De acordo com a NBR6024 as palavras-chaves devem ser separadas entre si por ponto e devem ter somente a primeira palavra escrita com letra maiúscula
% \setlength{\absparsep}{18pt} % ajusta o espaçamento dos parágrafos do resumo
\begin{resumo}
	\text Este trabalho foi desenvolvido com o intuito de demonstrar a funcionalidade de uma aplicação de conversa através de uma rede, aplicando conceitos de comunicação em rede. Para este objetivo, foi apresentado uma introdução a diversos conceitos fundamentais ao entendimento da aplicação e transferência de dados através de uma rede.
	\text No decorrer deste documento, será detalhado as definições e padrões da Internet, os diferentes protocolos de comunicação e também os diferentes tipos de topologia de redes de comunicação.
	\vspace{\onelineskip}

	\noindent
	\textbf{Palavras-chaves}: Internet. Protocolos. Padrões da Internet. Redes. Topologia de Redes.
\end{resumo}

% ----------------------------------------------
% Resumo em inglês
% ----------------------------------------------
% Importante: De acordo com a NBR6024 as palavras-chaves devem ser separadas entre si por ponto e devem ter somente a primeira palavra escrita com letra maiúscula
\begin{resumo}[Abstract]
	\begin{otherlanguage*}{english}
		\text This document was developed within the intention of demonstrating the functionality of a local network chat application, applying concepts of network communication. To this objective, it was presented an introduction to various fundamental concepts to the understanding of the application and transfer of data through a network.
		\text Through this work, it will be detailed the definitions and standarts of the internet, the different protocols of communication aswell as many different kinds of communication network topologies.
		\vspace{\onelineskip}

		\noindent
		\textbf{Keywords}: Internet. Protocols. Internet Standarts. Networks. Network Topologies.
	\end{otherlanguage*}
\end{resumo}

% ----------------------------------------------
% inserir lista de ilustrações
% ----------------------------------------------
% \pdfbookmark[0]{\listfigurename}{lof}
% \listoffigures*
% \cleardoublepage

% Diferentes tipos de listas podem ser criadas por meio de macros do memoir.

% ----------------------------------------------
% inserir lista de tabelas
% ----------------------------------------------
% \pdfbookmark[0]{\listtablename}{lot}
% \listoftables*
% \cleardoublepage

% ----------------------------------------------
% inserir lista de abreviaturas e siglas
% ----------------------------------------------
% Importante: As abreviaturas e siglas devem estar em ordem alfabética
% \begin{siglas}
%   \item[ABNT] Associação Brasileira de Normas Técnicas
%   \item[abnTeX] ABsurdas Normas para TeX
% \end{siglas}

% ----------------------------------------------
% inserir lista de símbolos
% ----------------------------------------------
% Importante: Os símbolos devem estar na ordem de aparecimento no texto.
% \begin{simbolos}
%   \item[$ \Gamma $] Letra grega Gama
%   \item[$ \Lambda $] Lambda
%   \item[$ \zeta $] Letra grega minúscula zeta
%   \item[$ \in $] Pertence
% \end{simbolos}

% ----------------------------------------------
% inserir o sumário
% ----------------------------------------------
\pdfbookmark[0]{\contentsname}{toc}
\tableofcontents*
\cleardoublepage
